\documentclass{X:/Documents/Coding/Latex/myreport}
\usepackage[toc,page]{appendix}
\title{Fluid Thesis}

%-----------------------------------------------------------------------
% theorems, lemma etc
\theoremstyle{plain}
\newtheorem{theorem}{Theorem}[section]
\newtheorem{lemma}[theorem]{Lemma}
\newtheorem{proposition}[theorem]{Proposition}
\newtheorem{question}[theorem]{Question}

\theoremstyle{definition}
\newtheorem{definition}[theorem]{Definition}

\theoremstyle{remark}
\newtheorem{example}{Example}[section]
\newtheorem{note}{Note}[section]
\newtheorem{exercise}{Exercise}[section]


\numberwithin{equation}{section}
\numberwithin{figure}{section}


\begin{document}

%
%-----beginning of title page -----------------------------
%
\begin{titlepage}
\begin{flushleft}
\hrule
\vspace{1 cm}


{\huge{\bf TITLE GOES \\[15pt] HERE}}
\vspace*{2cm}




\vspace{1 cm}
{\large S. T. Udent}

\vspace{0.5 cm}

{\large Supervisor: Professor Bloggs}

\vspace{0.5 cm}

{October 20??}

\vspace{2.5 cm}

{ Thesis submitted for the degree of Honours in ?????}



\vspace{10 cm}


\hrule


\end{flushleft}
\begin{flushleft}
\textbf{\textsf{SCHOOL OF MATHEMATICAL SCIENCES}}
\end{flushleft}


\vspace{-1cm}
%
% Adelaide University Crest
%

%
\begin{flushright}
\includegraphics[scale=0.25]{ua_crest.pdf}
\end{flushright}
\vspace{-2 cm}

\end{titlepage}


%
%---------- end of title page -------------------------------
%
\pagenumbering{roman}

\chapter*{Declaration}

Except where stated this thesis is, to  the best of my knowledge,  my own work and my supervisor has approved its submission.

\vspace{20 pt}

\begin{flushleft}
Signed by student:  \\[15 pt]
Date:
\end{flushleft}

\vspace{20 pt}
\begin{flushleft}
Signed  by supervisor:\\[15 pt]
Date:
\end{flushleft}


\chapter*{Acknowledgements}

\chapter*{Abstract}

\begin{abstract}
	Abstract
\end{abstract}

\tableofcontents

\pagenumbering{arabic}
%
%---------------- thesis ------------------------------------
%
\chapter{Introduction}


\section{Theory}
%definition, explanation of what a stream function is
%theorems, form in cylindrical coords
\definition{Stream Function}
\theorem{}
\lemma

\chapter{The Squire-Long model}
%outline of squirelong derivation
%key assumptions

%navier stokes background

\chapter{Flow Examples}
%Here we explore some different flow types and their behaviour
%Should describe what the flow is, and where we might see it
%Particularly talk about the assumptions we will universally make
%E.g. $\Psi(r) = \frac12 Wr^2$ initially
\section{Rotating Flow}
\section{Rankine Vortex}
\section{Lamb-Oseen Vortex}

% \chapter{}
\chapter{Solver}

\section{Computational Efficiency}


\begin{appendices}
\chapter{Derivation of Squire-Long}
%The actual full derivation here?
\chapter{Code}
\section{Code1}
	
\end{appendices}


















\begin{thebibliography}{9}
%
% the 9 here is so LaTeX knows how  big a space to leave for numbers
% in the bibliography. Put 9 if you have < 10 items, 99 if you have
% less than 100 etc
%


% \bibitem{Hit}
% N.J. Hitchin, Lectures on special Lagrangian 
% submanifolds. Preprint, Oxford 1999.



% \bibitem{Mur}
% M. K. Murray, 
% Bundle gerbes, 
% J. London Math. Soc. (2) {\bf 54} 
% (1996), no.~2, 403--416.


\end{thebibliography}
\end{document}

